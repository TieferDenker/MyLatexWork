\documentclass[12pt]{article}
\usepackage{amssymb}
\usepackage{amsfonts}
\usepackage{amsmath}
\usepackage{commath}
\usepackage{csquotes}
\usepackage{enumitem}
\usepackage{graphicx}
\usepackage{mathtools}
\usepackage{latexsym}
%\usepackage{txfonts} %darkens text
\usepackage[utf8]{inputenc}
\usepackage{wasysym}

\begin{document}
\begin{center}\Large \bf{Amazing Proofs: Episode 10 }\end{center}

\section*{Aim:}
In this episode of \enquote{Amazing Proofs}, we are going to derive Rogers-Ramanujan identities.

\section*{Statement:}
If $|q|<1$, then,
$$1+\sum_{n=1}^{\infty}\frac{q^{n^2}}{(q;q)n}=\frac{1}{(q;q^5){\infty}(q^4;q^5)_{\infty}}$$
and 
$$1+\sum_{n=1}^{\infty}\frac{q^{n^2+n}}{(q;q)n}=\frac{1}{(q^2;q^5){\infty}(q^3;q^5)_{\infty}}$$

\section*{Materials Required:}
Basics of $q$-series, Series and product manipulation, Jacobi Triple Product Identity (JTPI proved in episode 4)

\section*{Prerequisites:}
Let us define a few functions first,
$$P_r=\prod_{s=1}^{r}\frac{1}{1-q^s}=\frac{1}{(q;q)r},\hspace{.2 cm} Q_r(a)=\prod{s=r}^{\infty}\frac{1}{1-aq^s},\hspace{.2 cm} \lambda(r)=\frac{r(5r+1)}{2}$$
Now let us define an operator $\eta$ as,
$$\eta f(a)=f(aq)$$
Finally, let us define a simple sum,
$$H_m(a)=\sum_{r=0}^{\infty}(-1)^ra^{2r}q^{\lambda(r)-mr}(1-a^mq^{2mr})P_rQ_r$$
where, $P_0=1$, $|a|<1$, $r\in\mathbb{N}\cup\{0\}$ and $m=0,1,2$.

\section*{Procedure:}
Let us start by first observing that,
\begin{equation}  \label{eq:1}
H_m-H_{m-1}=\sum_{r=0}^{\infty}(-1)^ra^{2r}q^{\lambda(r)}P_rQ_rC_{mr}
\end{equation}
where,
$$C_{mr}=\frac{1-a^mq^{2mr}}{q^{mr}}-\frac{1-a^{m-1}q^{2(m-1)r}}{q^{(m-1)r}}$$
which can be simplified to,
$$C_{mr}=a^{m-1}q^{r(m-1)}(1-aq^r)+q^{-mr}(1-q^r)$$
observe again that,
$$(1-aq^r)Q_r(a)=Q_{r+1}(a)=Q_{r+1}$$
and 
$$(1-q^r)P_r=P_{r-1}$$
thus, $(\ref{eq:1})$ becomes,
$$H_m-H_{m-1}=\sum_{r=0}^{\infty}(-1)^ra^{2r}q^{\lambda(r)}(a^{m-1}q^{r(m-1)}P_rQ_{r+1}+q^{-mr}P_{r-1}Q_{r})$$
since, $P_0=1$, we have by definition, $P_{-1}=0$, therefore,
$$H_m-H_{m-1}=\sum_{r=0}^{\infty}(-1)^ra^{2r+m-1}q^{\lambda(r)+r(m-1)}P_rQ_{r+1}+\sum_{r=1}^{\infty}(-1)^ra^{2r}q^{\lambda(r)-mr}P_{r-1}Q_{r}$$
replacing $r$ by $r+1$ in the second summation, we have,
\begin{equation}  \label{eq:2}
H_m-H_{m-1}=\sum_{r=0}^{\infty}(-1)^rP_rQ_{r+1}D_{mr}
\end{equation}
where,
$$D_{mr}=a^{2r+m-1}q^{\lambda(r)+r(m-1)}-a^{2(r+1)}q^{\lambda(r+1)-m(r+1)}$$
since $\lambda(r+1)-\lambda(r)=5r+3$, the above equation can be written as,
$$D_{mr}=a^{2r+m-1}q^{\lambda(r)+r(m-1)}(1-a^{3-m}q^{(2r+1)(3-m)})$$
using the definition of $\eta$, we obtain,
$$D_{mr}=a^{m-1}\eta(a^{2r}q^{\lambda(r)-r(3-m)}(1-a^{3-m}q^{2r(3-m)}))$$
since $\eta Q_r=Q_{r+1}$, $(\ref{eq:2})$ becomes,
$$H_m-H_{m-1}=\sum_{r=0}^{\infty}(-1)^rP_r(\eta Q_r)a^{m-1}\eta(a^{2r}q^{\lambda(r)-r(3-m)}(1-a^{3-m}q^{2r(3-m)}))$$
which can be written as,
$$H_m-H_{m-1}=a^{m-1}\eta\Bigg(\sum_{r=0}^{\infty}(-1)^rP_r Q_ra^{2r}q^{\lambda(r)-r(3-m)}(1-a^{3-m}q^{2r(3-m)})\Bigg)$$
therefore,
$$H_m-H_{m-1}=a^{m-1}\eta H_{3-m}$$
Substituting $m=1$ and $m=2$ in the above equation, we have,
$$H_1-H_0=\eta H_2,\hspace{.2 cm}H_2-H_1=a\eta H_1$$
since $H_0=0$, adding both of them, we obtain,
$$H_2-H_0=\eta H_2+a\eta H_1\implies H_2=\eta H_2+a\eta^2 H_2$$
Now, let us now expand $H_2(a)$ in terms of the variable $a$, thus,
\begin{equation}  \label{eq:3}
H_2(a)=\sum_{n=0}^{\infty}c_na^n
\end{equation}
where, $c_0=1$. Using the equation $H_2=\eta H_2+a\eta^2 H_2$, we have,
$$\sum_{n=0}^{\infty}c_na^n=\sum_{n=0}^{\infty}c_na^nq^n+\sum_{n=0}^{\infty}c_na^{n+1}q^{2n}$$
comparing the coefficients of $a^n$, we obtain,
$$c_n=c_nq^n+c_{n-1}q^{2n-2}\implies c_n=\frac{q^{2n-2}}{1-q^n}c_{n-1} $$
solving this recurrence relation, we obtain,
$$c_n=\frac{q^{2n-2}}{1-q^n}\frac{q^{2n-4}}{1-q^{n-1}}\frac{q^{2n-6}}{1-q^{n-2}}...\frac{q^2}{1-q^2}\frac{q^0}{1-q}c_0=\frac{q^{n(n-1)}}{(q;q)_n}=q^{n(n-1)}P_n$$
finally, $(\ref{eq:3})$ becomes,
$$H_2(a)=\sum_{n=0}^{\infty}q^{n(n-1)}P_na^n$$
using the definition of $H_2(a)$ (from the \enquote{Prerequisites} section), we have,
$$H_2(a)=\sum_{r=0}^{\infty}(-1)^ra^{2r}q^{\lambda(r)-2r}(1-a^2q^{4r})P_rQ_r$$
substituting $a=q$ in the above equations, we obtain,
$$H_2(q)=1+\sum_{n=1}^{\infty}\frac{q^{n^2}}{(q;q)n}=\sum{r=0}^{\infty}(-1)^rq^{\lambda(r)}(1-q^{4r+2})P_rQ_r(q)$$
Let us simplify the second sum in the above equation by first observing that,
$$P_rQ_r(q)=\prod_{s=1}^{r}\frac{1}{1-q^s}\prod_{s=r}^{\infty}\frac{1}{1-q^{s+1}}=\prod_{s=1}^{\infty}\frac{1}{1-q^s}=P_{\infty}=\frac{1}{(q;q)_{\infty}}$$
the remaining part inside the summation can be written as,
$$\sum_{r=0}^{\infty}(-1)^rq^{\lambda(r)}(1-q^{4r+2})=1+\sum_{r=1}^{\infty}(-1)^rq^{\lambda(r)}-\sum_{r=1}^{\infty}(-1)^rq^{\lambda(r-1)+4(r-1)+2}$$
which can be simplified further as,
$$\sum_{r=0}^{\infty}(-1)^rq^{\lambda(r)}(1-q^{4r+2})=1+\sum_{r=1}^{\infty}(-1)^r\Bigg(q^{\frac{r(5r+1)}{2}}+q^{\frac{r(5r-1)}{2}}\Bigg)$$
thus,
\begin{equation}  \label{eq:4}
H_2(q)=\frac{1}{(q;q){\infty}}\Bigg(1+\sum{r=1}^{\infty}(-1)^r\Bigg(q^{\frac{r(5r+1)}{2}}+q^{\frac{r(5r-1)}{2}}\Bigg)\Bigg)
\end{equation}
It is known from JTPI that,
$$(q^2;q^2){\infty}(-zq;q^2){\infty}(-z^{-1}q;q^2){\infty}=\sum{n=-\infty}^{\infty}q^{n^2}z^n$$
replacing $q$ by $q^{5/2}$ and $z$ by $-q^{1/2}$ in JTPI, we obtain,
$$(q^5;q^5){\infty}(q^3;q^5){\infty}(q^2;q^5){\infty}=\sum{n=-\infty}^{\infty}q^{5n^2/2}(-q)^{n/2}=1+\sum_{n=1}^{\infty}(-1)^n\Bigg(q^{\frac{n(5n+1)}{2}}+q^{\frac{n(5n-1)}{2}}\Bigg)$$
therefore, $(\ref{eq:4})$ becomes,
$$H_2(q)=\frac{(q^2;q^5){\infty}(q^3;q^5){\infty}(q^5;q^5){\infty}}{(q;q){\infty}}=\frac{1}{(q;q^5){\infty}(q^4;q^5){\infty}}$$
this completes the proof of the first identity.\\
We will follow the same lines of proof to prove the second identity, observe that,
$$H_1=\eta H_2=\sum_{n=0}^{\infty}q^{n^2}P_na^n$$
using the definition of $H_1(a)$ (from the “Prerequisites” section), we have,
$$H_1(a)=\sum_{r=0}^{\infty}(-1)^ra^{2r}q^{\lambda(r)-r}(1-aq^{2r})P_rQ_r$$
substituting $a=q$ in the above equations and replacing $q$ by $q^{5/2}$ and $z$ by $-q^{3/2}$ in JTPI, we complete the proof of the second identity.

\section*{Corollary:}
In the next episode, we will derive Rogers-Ramanujan continued fraction.

\section*{Conclusion:}
If we interpret the above identity combinatorially, we conclude that the number of partitions of $n$ such that the adjacent parts differ by at least 2 is the same as the number of partitions of $n$ such that each part is congruent to either 1 or 4 modulo 5. Furthermore these identities are used in forming the well known Rogers-Ramanujan continued fraction and various other beautiful identities related to modular forms which will be discussed in the later episodes.
\end{document}